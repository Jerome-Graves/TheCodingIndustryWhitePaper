\chapter{Current Technology}
\pagestyle{fancy}

There are two main over-printing technologies:

\begin{itemize}
    \item Thermal transfer
    \item Inkjet
  \end{itemize}

  \section{Inkjet}

  Inkjet printers use liquid ink in cartridges, and a  mechanical mechanism to move the nozzle across the substrate whiles the substrate is moved perpendicularly underneath. As such, Ink-jets are mostly continuous printers, although there are some intermittent variations.

  The main issue with ink-jets is the use of expensive liquid ink, which needs to be replaced often and can leak, causing massive production line issues. The quality can be mixed as time is needed for the ink to dry and smudging can occur. The level of smudging depends on the material printed on with less porous materials being worse. Also, printing must be done in a flat position, machines cant print inverted or at a steep angle. This creates limitations when installing the machines.


  \section{Thermal transfer}

  Thermal transfer printers use ink with wax on a roll of thin plastic ribbon. The machine unrolls this as ink is needed. To deposit the ink, the printhead presses the ribbon against the substrate and uses heater elements to melt and transfer the ink-wax mixture. 
  The printhead contains hundreds of heater elements in a horizontal line, each corresponding to a pixel of information.
  In intermittent printing, the printhead moves across the substrate using a mechanical mechanism activating different heater elements across the ribbon to transfer ink. 
  In continuous operation, the printhead synchronises its elements with the moving substrate underneath to deliver ink. This allows for print while inverted and at steep angles. There is very good ink adhesion with plastic materials. 
  
  \section{Hot Foil}

  Before thermal transfer in ink-jet, hot foil machines were the majority and they are still used today. The machine uses a similar process as a thermal transfer printer but uses a heated stamp instead of the elements. Every printed image used a different stamp configuration.

\begin{mdframed}[style=custom,frametitle=\colorbox{berkeleygold}{\space Put your title here \space}]

You might like to use these boxes for essays or to break up text. Unlike the mybox environment, mdframed will break across pages.

Don't like the colors?

You can change them by editing the sidebar section of report.tex. In mdfdefinestyle, look for \textbackslash colorbox\{berkeleygold\} to alter the title background. If you want no color, just change berkeleygold to white. If you want to dilute the color a bit, you could add a !30 to the end of berkeleygold, as in berkeleygold!30

The background color is set by backgroundcolor and it is currently founderblue!20. You can deepen it by increasing the number or lighten it by reducing the number to 10 or even 5.

\end{mdframed}


\begin{mdframed}[style=custom,frametitle=\colorbox{berkeleygold}{\space This is my second box \space}]

Please note, you have to specify your title in each box in the frametitle option. Just put your title between the \textbackslash space commands.

\end{mdframed}

