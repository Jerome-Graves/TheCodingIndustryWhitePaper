\chapter{Future Technologies}
\pagestyle{fancy}

\section{Laser markers and engravers}
Laser engraving/cutting and marking technology have been around for over 50 years. Generally, it is used on ceramics, wood and metals. They work by moving a high-powered laser over the surface of the substrate. The surface is burnt leaving an irremovable mark.

\section{Scanning lasers}
A scanning laser uses a set of small mirrors on rotating arms, known as a galvo, to direct its beam. Due to the mirrors' low mass, extremely fast movement can be attained. 

\section{High-frequency UV Laser Markers}
UV laser uses a different wavelength than standard laser markers and can be used on plastic materials. By modifying the pulse width and frequency, their burn effect can be minimised and instead activates annealing effects on the substrate, causing different contrast effects.

\section{Subtractive marking}
When dealing with thin, delicate materials etching the surface may cause excessive deformation. Subtractive marking is then used. The laser is configured to remove the ink from a darkened area on the packaging. All packaging is pre-printed with logos and general information externally before it reaches the production line. 
Adding a darkened square on each pack results in a minimal cost increase( If any), as the machines used in this printing process are more ink efficient.


\begin{tcolorbox}[breakable,title={An example colorbox},
colback=founderblue!5!white,
colframe=founderblue!75!black,
fonttitle=\headingfont\bfseries\large]
Your text goes here. The colors are based on Berkeley's Founder Blue.
\tcbsubtitle[before skip=\baselineskip]%
{You can also have subboxes}
Don't like the colors? You can change them here (replace founderblue with the color you like), or you can change them globally in the colors section of report.tex
\tcbsubtitle[before skip=\baselineskip]%
{Multiple rows}
If you like it.
\end{tcolorbox}






