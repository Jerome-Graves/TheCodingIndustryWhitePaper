\chapter{Introduction}
\pagestyle{fancy}

Coding machines are an integral part of the manufacturing industry and are an important part of how governments regulate products. The manufacturing industry grew by 8.3\% and is predicted to increase in the coming years.
As production output increases, overall production speeds will increase to meet the demand. This is directly proportional to the coding industry growth. Whether it be a sell-by date, barcode or an EAN number, products are legally required to be marked either in batches or individually. 

\bigskip
This job has traditionally been done by \textbf{over-printers}.


\bigskip

\section{What is an over-printer?}

Rather than a classic printer that feeds the material into its system, an over-printer is a machine designed to print directly on the material, from above (or other angles). The main benefit of this is the ability to mark products at different points with little interference in the assembly line.

\subsection{Intermittent and continuous printers?}

There are two main types of operation for over-printers, intermittent and continuous. They refer to the operation of the parent machine (the machine the printer is attached to). Some production lines operate in a stop-start fashion giving time to complete batch tasks before moving products to the next position. Other production lines work without stopping.

\begin{itemize}
    \item \textbf{Intermittent} printers are required if the production line stops and starts. They print on the substrate(material to be printed on) when it is stationary. They print on a flat surface with a mechanical arm in the printer that moves the printhead across the substrate.
    \item \textbf{Continuous} printers are used when the substrate does not stop. They print on the substrate as it is traveling. They usually print on a roller (see below). The substrate is moved by the parent machine (the machine the printer is attached to). The printing speed is also synchronized with the parent machine.
\end{itemize}

\subsection{Prints per minutes (PPM) Vs Print speed(mm/s)?}

A few values are used to show the speed of a printer. The most common are PPM and Print speed. The PPM is the maximum number of prints the machine can do in a minute. The print speed is the speed at which ink is transferred to the substrate. It is hard to draw any meaning from PPM  as manufacturers do not refer to the size of the printed image. Print speed better represents the printer's capability for high production as PPM can be calculated situationally.


\subsection{Left and right-hand printers?}

Most continuous over printers only print on the substrate in a given direction. Due to the variety of parent machines and the fact that cartridges need to be easy accesses, some companies produce mirror printers that operation the opposite direction.
