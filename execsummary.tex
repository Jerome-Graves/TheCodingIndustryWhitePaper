\chapter*{Executive Summary}
\pagestyle{fancy}

Coding is an integral part of manufacturing. In every country in the world, there are strict rules for individual marking for products. Every product sold in a marketplace requires a barcode, even if you make it yourself. These can be \textbf{sell-by dates}, \textbf{UPC} or \textbf{EAN} numbers.


\bigskip
A  manufacturing company must have systems in place to individually mark items. These markings must have a level of visual fidelity to regional or international standards. And they should operate at a speed that doesnt slow doesnt production. As global demand has increased, the manufacturing industry has met the challenge with an increase in production (8.3percent in 2021). Marking technology has seen few improvements over the last 30 years and may soon become a bottleneck in production.

\bigskip
Most of the coding industry uses over-printers to complete coding, although conventional printing techniques are still used. Many commercially available manufacturing machines have hardpoint connections for the industry-leading coding printers. 

\bigskip
The main coding printing technologies are Dot Matrix, Inkjet, Laser and Thermal printers. Dot matrix creates codes that can be difficult to scan. Laser printers can only print conventionally and cannot over-print, making them unusable in many scenarios. Inkjet also has limitations in how it can be placed and used. Printing at steep angles or inverted is not possible. The introduction of wet ink also can cause havoc on a production line. Direct thermal requires thermal material to print on, usually thermal paper. This can cause issues as it needs to be transferred to the product, usually for a sticker label. These labels can be removed from products which is a less effective solution. 

\bigskip
By far the best and most used method of coding products is thermal transfer printing. Their printing method allows for the direct marking of many types of packaging materials, with high speeds and quality at an adorable price.

\bigskip
Although these printing machines have slightly different techniques, they all suffer from essentially the same limitations. 

\begin{itemize}
    \item Reliability/Durability
    \item High cost of consumables Ink, printheads
\end{itemize}

The mechanical nature of transferring ink to the material's surface means that these printers have many parts that can become worn over time, as well as physical limits to their achievable speed. As factory production speeds increase the prints per minute of the printerPPM increase and their consumable parts reach the end of their life much faster. This means partially stopping production by swapping the ink medium or a printhead.

\bigskip
Ink can also come in many forms but generally is shipped in small packets, where a large portion of the mass of the cartridge is plastic packaging. The most used technology currently (thermal transfer printing) has a 60-80percent ink transfer efficiency. These factors can have a significant effect on product margins. 

\bigskip
High-frequency UV laser marking is an advancing technology that addresses many of these issues. Laser making is a robust technology that has been used to mark packaging and products for over 30 years. It is an inkless technology that burns irremovable marks into the item. Its main limitation is that it can only be used on certain materials due to the high play nature of the laser?

\bigskip
UV lasers have less energy and work well on plastics. Also with High-frequency pulsing, there is greater control of the laser power, allowing for annealing effects on plastics. This inkless process doesn't require contact, reducing cost and dramatically increasing reliability. It is also faster than traditional printing with the slow machine being capable of 600/m/s, which is at the top end for overprinting. It is also far more accurate allowing for double the dpi (dots per inch) of a traditional over-printer. 




% The Asterix after chapter here excludes executive summary from the table of content and ensures it doesn't get a number (e.g. chapter 1)



%\section*{Introduction}
% The Asterix after section causes it to not be included in the table of contents

%\lipsum[1]

%\subsection*{Sub section title for exec summary}

%\lipsum[2-3]
