\chapter{Technological Limitations}
\pagestyle{fancy}

\section{Speed Limitations}
The speed limitation of an over-printer is determined by its mechanical design, the properties of the substrate and the ink used. The limitations with mechanical design can mostly be overcome without limit, but the chemical properties of currently available inks seem to be the industry's major limiting factor. 
There are over-printers that boast speeds of around 1m/s and a few special examples of even faster speeds of 1500m/s that require a particular substrate and ink combination, but in a future where production lines will have certain elements working  2-4m/s, the over-printer will become a bottleneck that can only be alleviated by buying multiple machines.

\section{Durability}
Over-printers have many mechanical elements. As production speed increases, over-printers will have to become larger and bulker.  The Mechanics that move the printhead at higher speeds will need to be more robust to counteract the momentum generated. They will use ink faster so they will need larger body sizes for ink storage.
Printheads can also be considered consumable they have a limited life and will need to be changed more often with increased production. Every time A printer ink needs to be changed the production line must stop (in that area).
These machines will reach a point where their size, cost and amount of human maintenance needed will make them an ineffective solution.

\section{Consumables}

As production speed increases so do the use of consumables. Most over-printer have warranties that require you to buy a particular brand of ink much like home printers. Unlike home printers, Industrial printers print continuously for up to 7 days a week. This generates a massive amount of usage that dwarfs the cost of the over-printer. An over-printer that is being used moderately can use £100,000 of ink in a year. As production speed increases, this cost will increase proportionally.
This cost is exacerbated by the fact that the machines don’t use 100% Ink in their cartridge. Depending on the style of the printer up to 50% of this will remain in the ink transmission medium depending on what is printed( thermal transfer printers).
Most inks are also sold on a transmission medium with a large amount of unrecyclable plastic, this also increases proportionally with production as well as shipping costs.





\begin{tcolorbox}[breakable,title={An example colorbox},
colback=founderblue!5!white,
colframe=founderblue!75!black,
fonttitle=\headingfont\bfseries\large]
Your text goes here. The colors are based on Berkeley's Founder Blue.
\tcbsubtitle[before skip=\baselineskip]%
{You can also have subboxes}
Don't like the colors? You can change them here (replace founderblue with the color you like), or you can change them globally in the colors section of report.tex
\tcbsubtitle[before skip=\baselineskip]%
{Multiple rows}
If you like it.
\end{tcolorbox}






